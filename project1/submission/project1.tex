\documentclass[twoside,11pt]{article}

\usepackage{aa228-jmlr2e}
\usepackage{lipsum}
\usepackage{listings}

\input{julia_listing}

\begin{document}

% Refer to this link for project rubric: https://aa228.stanford.edu/project-1/
\title{Project 1: Bayesian Structure Learning}

%===========================================
% TODO: Replace "First Last" with your name.
% TODO: Replace "email@stanford.edu" with your Stanford email.
%===========================================
\name{First Last}
\email{email@stanford.edu}


\maketitle


\section{Algorithm Description}
%===========================================
% TODO: Replace this with a short description of your algorithm(s) used.
\lipsum[2]
%===========================================



\section{Graphs}
%===========================================
% TODO: Add your small, medium, and large graph visualizations here
%===========================================
\begin{figure}[h]
    \centering
    \includegraphics[width=0.4\textwidth]{example_graph.pdf}
    \caption{Graph caption.}
\end{figure}



\section{Code}
%===========================================
% TODO: Add your code here, see code listing options here: https://www.overleaf.com/learn/latex/code_listing
% NOTE: Code does not count towards your page limit!
% OPTIONS:
%   1. Paste everything into a {verbatim} environment (where all characters are parsed...verbatim).
%   or 2. paste everything into a {lstlisting} environment for syntax highlighting (examples for Julia and Python below).
% NOTE: Feel free to break up functions into separate {algorithm} + {lstlisting} environments for better organization (not required!) 
%===========================================


%===========================================
% EXAMPLE JULIA: TODO REPLACE WITH YOUR CODE
%===========================================
\begin{algorithm}
\begin{lstlisting}[language=Julia]
using LightGraphs
using Printf

"""
    write_gph(dag::DiGraph, idx2names, filename)

Takes a DiGraph, a Dict of index to names and a output filename to write the graph in `gph` format.
"""
function write_gph(dag::DiGraph, idx2names, filename)
    open(filename, "w") do io
        for edge in edges(dag)
            @printf(io, "%s,%s\n", idx2names[src(edge)], idx2names[dst(edge)])
        end
    end
end


function compute(infile, outfile)

    # WRITE YOUR CODE HERE
    # FEEL FREE TO CHANGE ANYTHING ANYWHERE IN THE CODE
    # THIS INCLUDES CHANGING THE FUNCTION NAMES, MAKING THE CODE MODULAR, BASICALLY ANYTHING

end

if length(ARGS) != 2
    error("usage: julia project1.jl <infile>.csv <outfile>.gph")
end

inputfilename = ARGS[1]
outputfilename = ARGS[2]

compute(inputfilename, outputfilename)
\end{lstlisting}
\end{algorithm}


%===========================================
% EXAMPLE PYTHON, TODO REPLACE WITH YOUR CODE:
%===========================================
\begin{algorithm}
\begin{lstlisting}[language=Python]
import sys

import networkx


def write_gph(dag, idx2names, filename):
    with open(filename, 'w') as f:
        for edge in dag.edges():
            f.write("{}, {}\n".format(idx2names[edge[0]], idx2names[edge[1]]))


def compute(infile, outfile):
    # WRITE YOUR CODE HERE
    # FEEL FREE TO CHANGE ANYTHING ANYWHERE IN THE CODE
    # THIS INCLUDES CHANGING THE FUNCTION NAMES, MAKING THE CODE MODULAR, BASICALLY ANYTHING
    pass


def main():
    if len(sys.argv) != 3:
        raise Exception("usage: python project1.py <infile>.csv <outfile>.gph")

    inputfilename = sys.argv[1]
    outputfilename = sys.argv[2]
    compute(inputfilename, outputfilename)


if __name__ == '__main__':
    main()

\end{lstlisting}
\end{algorithm}


\end{document}